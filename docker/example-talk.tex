\documentclass[12pt]{beamer}

\usetheme{Oxygen}
\usepackage{thumbpdf}
\usepackage{wasysym}
\usepackage{ucs}
\usepackage[utf8]{inputenc}
\usepackage{pgf,pgfarrows,pgfnodes,pgfautomata,pgfheaps,pgfshade}
\usepackage{verbatim}

\pdfinfo
{
  /Title       (Introdução ao Docker)
  /Creator     (Thiago Almeida)
  /Author      (Thiago Almeida)
}


\title{Introdução ao Docker}
\subtitle{O que é e porque devo utilizar?}
\date{\today}

\begin{document}
\author{Thiago Almeida
    	\href{mailto:thiagoalmeidasa@gmail.com}{thiagoalmeidasa@gmail.com}}

\frame{\titlepage}

%\section*{}
%\begin{frame}
%  \frametitle{Outline}
%  \tableofcontents[section=1,hidesubsections]
%\end{frame}
%
%\AtBeginSection[]
%{
%  \frame<handout:0>
%  {
%    \frametitle{Outline}
%    \tableofcontents[currentsection,hideallsubsections]
%  }
%}
%
%\AtBeginSubsection[]
%{
%  \frame<handout:0>
%  {
%    \frametitle{Outline}
%    \tableofcontents[sectionstyle=show/hide,subsectionstyle=show/shaded/hide]
%  }
%}

\newcommand<>{\highlighton}[1]{%
  \alt#2{\structure{#1}}{{#1}}
}

\newcommand{\icon}[1]{\pgfimage[height=1em]{#1}}



\newcommand{\putlink}[1]{%
   \pgfsetlinewidth{1.4pt}%
   \pgfsetendarrow{\pgfarrowtriangle{4pt}}%
   \pgfline{\pgfxy(1,1)}{\pgfxy(#1,1)}
}
%%%%%%%%%%%%%%%%%%%%%%%%%%%%%%%%%%%%%%%%%
%%%%%%%%%% Content starts here %%%%%%%%%%
%%%%%%%%%%%%%%%%%%%%%%%%%%%%%%%%%%%%%%%%%



\section{Apresentação}

\begin{frame}
  \frametitle{Pré requisitos e Objetivos}
  \begin{block}{Docker}
  \begin{itemize}
		\item Nenhuma experiência anterior com \emph{Docker} será necessária.
  \end{itemize}
  \end{block}

	\begin{block}{\emph{LINUX}}
  \begin{itemize}
		\item Familiariadade com alguma distribuição \emph{Linux}
		\item Familiariadade com terminal e linhas de comando.
  \end{itemize}
  \end{block}

  \begin{block}{Objetivos}
  \begin{itemize}
		\item Entender a a estrutura da plataforma \emph{Docker}.
		\item Conhecer os componentes da plataforma \emph{Docker}.
						\begin{itemize}
										\item Imagens
										\item Contêineres
										\item Repositórios
						\end{itemize}
  \end{itemize}
  \end{block}
\end{frame}

\section{Introdução}
\begin{frame}
  \frametitle{Agenda}
  \framesubtitle{Tópicos abordados}
  \begin{block}{Principais pontos}
					\begin{itemize}
									\item <2->O que é Docker?
									\item <3->Contêineres vs Máquinas Virituais
									\item <4->Visualização sobre a plataforma Docker
													\begin{itemize}
																	\item <5->Docker Engine
																	\item <6->Imagens
																	\item <7->Contêineres
																	\item <8->Registro
																	\item <9->Repositórios
																	\item <10->Docker Hub
																	\item <11->Ferramentas de orquestração Docker
													\end{itemize}
									\item <12->Introdução às Imagens
									\item <13->Iniciando com contêineres
  				\end{itemize}
  \end{block}
\end{frame}

\begin{frame}
  \frametitle{O que é Docker?}
  \begin{block}{Definição}
					Docker é uma plataforma para desenvolvimento, envio e execução de
					aplicações utilizando virtualização baseada em contêiner.
					A plataforma \emph{Docker} é composta por algumas ferramentas e
					produtos, são eles:
  \begin{itemize}
					\item <2->\emph{Docker Engine}
					\item <3->\emph{Docker Hub}
					\item <4->\emph{Docker Machine}
					\item <5->\emph{Docker Swarm}
					\item <6->\emph{Docker Compose}
  \end{itemize}
  \end{block}
\end{frame}

\begin{frame}
  \frametitle{Contêineres vs Máquinas Virtuais}
  \frametitle{Qual a diferença entre elas?}
  \begin{block}{Definição}
					Docker é uma plataforma para desenvolvimento, envio e execução de
					aplicações utilizando virtualização baseada em contêiner.
					A plataforma \emph{Docker} é composta por algumas ferramentas e
					produtos, são eles:
  \end{block}
\end{frame}

\section{Fancy features}

\section*{}
\frame{
  \vfill
  \centering
  \highlighton{
  \usebeamerfont*{frametitle}And now?

  \usebeamerfont*{framesubtitle}Enter the secret section
  }
  \vfill
}
\begin{frame}
  \frametitle{Contributing to this beamer style}
  \framesubtitle{We want you !}

  \begin{block}{Why?}
  \begin{itemize}
    \item Beamer is hot!
    \item This style deserves to be improved
  \end{itemize}
  \end{block}

  \begin{block}{How?}
  \begin{itemize}
    \item Grab it
    \item Improve its LaTeX code
    \item Use you artistics skills
    \item Document it
    \item Help other people to use it
    \item Use it...
  \end{itemize}
  \end{block}
\end{frame}

\begin{frame}
  \frametitle{Resources}
  \framesubtitle{If you want to improve this style}
  \begin{thebibliography}{10}

  \beamertemplatearticlebibitems

  \bibitem{beamer-homepage}
    LaTeX Beamer
    \newblock {\tt http://latex-beamer.sourceforge.net/}

  \bibitem{kdeslides}
    KDE Presentations
    \newblock {\tt http://www.kde.org/kdeslides/}

  \end{thebibliography}
\end{frame}

\frame{
  \vspace{2cm}
  {\huge Questions ?}

  \vspace{3cm}
  \begin{flushright}
    Konqi Konqueror

    \structure{\footnotesize{konqi@kde.org}}
  \end{flushright}
}

\end{document}
