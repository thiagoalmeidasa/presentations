\documentclass[12pt]{beamer}

\usetheme{Oxygen}
\usepackage{thumbpdf}
\usepackage{wasysym}
\usepackage{ucs}
\usepackage[utf8]{inputenc}
\usepackage{pgf,pgfarrows,pgfnodes,pgfautomata,pgfheaps,pgfshade}
\usepackage{verbatim}
\usepackage{graphicx}

\pdfinfo
{
  /Title       (Introdução ao Docker)
  /Creator     (Thiago Almeida)
  /Author      (Thiago Almeida)
}


\title{Introdução ao Docker}
\subtitle{O que é e porque devo utilizar?}
\date{\today}

\begin{document}
\author{Thiago Almeida
      \href{mailto:thiagoalmeida@ufpa.br}{thiagoalmeida@ufpa.br}}

\frame{\titlepage}

\newcommand<>{\highlighton}[1]{%
  \alt#2{\structure{#1}}{{#1}}
}

\newcommand{\icon}[1]{\pgfimage[height=1em]{#1}}



\newcommand{\putlink}[1]{%
   \pgfsetlinewidth{1.4pt}%
   \pgfsetendarrow{\pgfarrowtriangle{4pt}}%
   \pgfline{\pgfxy(1,1)}{\pgfxy(#1,1)}
}
%%%%%%%%%%%%%%%%%%%%%%%%%%%%%%%%%%%%%%%%%
%%%%%%%%%% Content starts here %%%%%%%%%%
%%%%%%%%%%%%%%%%%%%%%%%%%%%%%%%%%%%%%%%%%



\section{Apresentação}
\begin{frame}
  \frametitle{Pré requisitos e Objetivos}
  \begin{block}{Docker}
  \begin{itemize}
    \item Nenhuma experiência anterior com \emph{Docker} será necessária.
  \end{itemize}
  \end{block}

  \begin{block}{\emph{LINUX}}
  \begin{itemize}
    \item Familiaridade com alguma distribuição \emph{Linux}
    \item Familiaridade com terminal e linhas de comando.
  \end{itemize}
  \end{block}

  \begin{block}{Objetivos}
  \begin{itemize}
    \item Entender a estrutura da plataforma \emph{Docker}.
    \item Conhecer os componentes da plataforma \emph{Docker}.
      \begin{itemize}
        \item Imagens
        \item Contêineres
        \item Repositórios
      \end{itemize}
  \end{itemize}
  \end{block}
\end{frame}
\section{Introdução}
\begin{frame}
  \frametitle{Agenda}
  \framesubtitle{Tópicos abordados}
  \begin{block}{Principais pontos}
    \begin{itemize}
      \item O que é Docker?
      \item Contêineres vs Máquinas Virtuais
      \item Visualização sobre a plataforma Docker
        \begin{itemize}
          \item Docker Engine
          \item Imagens
          \item Contêineres
          \item Registro
          \item Repositórios
        \end{itemize}
      \item Introdução às Imagens
      \item Iniciando com contêineres
        \end{itemize}
  \end{block}
\end{frame}
\begin{frame}
  \frametitle{O que é Docker?}
  \begin{block}{Definição}
          Docker é uma plataforma para desenvolvimento, envio e execução de
          aplicações utilizando virtualização baseada em contêiner.
          A plataforma \emph{Docker} é composta por algumas ferramentas e
          produtos, são eles:
  \begin{itemize}
    \item \emph{Docker Engine}
    \item \emph{Docker Hub}
    \item \emph{Docker Machine}
    \item \emph{Docker Swarm}
    \item \emph{Docker Compose}
  \end{itemize}
  \end{block}
\end{frame}
\begin{frame}
  \frametitle{Contêineres vs Máquinas Virtuais}
  \framesubtitle{Um pouco de história}
  \begin{block}{Servidores reais}
    Antigamente nós utilizávamos um servidor para uma única aplicação.
    \begin{figure}[!h]
      \centering
      \includegraphics[width=0.6\paperwidth]{physicalserver}
    \end{figure}
  \end{block}
\end{frame}
\begin{frame}
  \frametitle{Contêineres vs Máquinas Virtuais}
  \framesubtitle{Um pouco de história}
  \begin{block}{Problemas que isso causava}
  \begin{itemize}
    \item Gastava muito tempo no \emph{deploy}
    \item Alto custo
    \item Desperdício de recursos
    \item Difícil para escalar
    \item Difícil para migrar
    \item Dependência do fabricante
  \end{itemize}
  \end{block}
\end{frame}
\begin{frame}
  \frametitle{Contêineres vs Máquinas Virtuais}
  \framesubtitle{Um pouco de história}
  \begin{block}{Virtualização baseada em \emph{Hypervisor}}
  \begin{itemize}
    \item Um servidor físico pode conter várias aplicações
    \item Cada aplicação roda em uma máquina virtual (\emph{VM})
    \begin{figure}[!h]
      \centering
      \includegraphics[width=0.6\paperwidth]{hypervisor}
    \end{figure}
  \end{itemize}
  \end{block}
\end{frame}
\begin{frame}
  \frametitle{Contêineres vs Máquinas Virtuais}
  \framesubtitle{Um pouco de história}
  \begin{block}{Vantagens da máquina virtual}
  \begin{itemize}
    \item Melhor aproveitamento dos recursos
    \item Um servidor físico dividido em várias máquinas virtuais
    \item Fácil de escalar
  \end{itemize}
  \end{block}
  \begin{block}{Limitações da máquina virtual}
    \begin{itemize}
      \item Cada máquina virtual requer:
        \begin{itemize}
        \item CPU alocada
        \item Armazenamento
        \item Memória RAM
        \item Um sistema operacional inteiro
      \end{itemize}
      \item Quanto mais \emph{VMs} você roda, mais recursos você precisa
      \item Um sistema operacional num \emph{guest} é um desperdício
    \end{itemize}
  \end{block}
\end{frame}
\begin{frame}
  \frametitle{Contêineres vs Máquinas Virtuais}
  \framesubtitle{Introdução aos contêineres}
  \begin{block}{O que são?}
          Virtualização baseada em contêiner usa o \emph{kernel} do sistema
          operacional do \emph{host} para executar múltiplas instâncias
  \end{block}
  \begin{block}{}
  \begin{itemize}
    \item Cada \emph{guest} é chamado de \textbf{contêiner}
    \item Cada contêiner possui:
      \begin{itemize}
        \item Sistema de arquivos raiz
        \item Processos
        \item Memória
        \item Dispositivos
        \item Pilha de rede
      \end{itemize}
  \end{itemize}
  \end{block}
\end{frame}
\begin{frame}
  \frametitle{Contêineres vs Máquinas Virtuais}
  \framesubtitle{Contêiner}
  \begin{block}{Visualizando um contêiner}
    \begin{figure}[!h]
      \centering
      \includegraphics[width=0.6\paperwidth]{container}
    \end{figure}
  \end{block}
\end{frame}
\begin{frame}
  \frametitle{Contêineres vs Máquinas Virtuais}
  \framesubtitle{Contêiner vs VM}
  \begin{block}{Vantagens do contêiner}
    \begin{itemize}
      \item Contêiner é mais leve
      \item Não precisa instalar um SO inteiro
      \item Requer menos \emph{CPU, RAM} e armazenamento 
      \item Um servidor pode rodar mais contêineres do que \emph{VMs}
      \item Maior portabilidade
    \end{itemize}
  \end{block}
\end{frame}
\begin{frame}
  \frametitle{Contêineres vs Máquinas Virtuais}
  \framesubtitle{Visualizando as diferenças}
  \begin{block}{}
    \begin{figure}[!h]
      \centering
      \includegraphics[width=0.6\paperwidth]{vmvscontainer}
    \end{figure}
  \end{block}
\end{frame}
\section{Docker, conceitos e termos}
\begin{frame}
  \frametitle{Docker Engine}
  \begin{block}{O que é?}
    \begin{itemize}
      \item \emph{Docker Engine} é o programa que possibilita os
                    contêineres serem feitos, enviados e executados.
      \item \emph{Docker Engine} utiliza \emph{namespaces e cgroups} do
              \emph{Kernel Linux}.
      \item \emph{Namespaces} nos permitem isolar os contêineres nos seus
              próprios ambientes.
    \end{itemize}
  \end{block}
\end{frame}
\begin{frame}
  \frametitle{Composição do Docker Engine}
  \begin{block}{Docker Client e Daemon}
    \begin{itemize}
      \item Possui a arquitetura Cliente/Servidor
      \item O cliente pega as entradas do usuário e às envia pro \emph{Daemon}
      \item O \emph{Daemon} monta, executa e distribui os contêineres.
      \item Cliente e \emph{Daemon} podem rodar no mesmo \emph{host} ou em
              \emph{hosts} diferentes.
    \end{itemize}
  \end{block}
\end{frame}
\begin{frame}
  \frametitle{Docker Contêineres e Imagens}
  \begin{block}{Imagens}
    \begin{itemize}
      \item \emph{Template} somente leitura utilizado para criar contêineres
      \item Feita por você mesmo ou outro usuários Docker
      \item Armazenada no \emph{Docker Hub} ou no seu \emph{Registro local}
    \end{itemize}
  \end{block}
  \begin{block}{Contêineres}
    \begin{itemize}
      \item Plataforma isolada para a aplicação
      \item Contém tudo que precisa para executar a aplicação
      \item Baseado em uma ou mais imagens
    \end{itemize}
  \end{block}
\end{frame}
\begin{frame}
  \frametitle{Docker Registro e Repositório}
  \begin{block}{Registro}
    \begin{itemize}
      \item Registro é onde nós armazenamos as nossas imagens.
      \item Podemos ter o nosso próprio Registro ou utilizar Registros públicos
              como o \emph{Docker Hub}
    \end{itemize}
  \end{block}
  \begin{block}{Repositórios}
    \begin{itemize}
      \item Dentro do Registro nós temos os Repositórios
      \item Cada Repositório armazena versões das imagens base
    \end{itemize}
  \end{block}
\end{frame}
\begin{frame}
  \frametitle{Docker Registro e Repositório}
  \begin{block}{Registro}
    \begin{figure}[!h]
      \centering
      \includegraphics[width=0.6\paperwidth]{registrorepo}
    \end{figure}
  \end{block}
\end{frame}
\end{document}
